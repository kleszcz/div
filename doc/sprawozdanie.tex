\documentclass[a4paper]{article}
\usepackage[utf8]{inputenc}

\author{Jan Kleszczyński i Michał Wielgus}
\title{PSC01: ,,Moduł dzielący binarnie dla liczb bez znaku''}
\date{2013/2014}

\usepackage{fullpage}
\usepackage{polski}
\usepackage{graphicx}
\usepackage{hyperref}
\usepackage[hypcap]{caption}
\usepackage{perpage}
\usepackage{listings}
\usepackage{amsmath}
\usepackage{multicol}
\usepackage{multirow}

\lstset{
	basicstyle=\footnotesize,
	breakatwhitespace=false,
	breaklines=true,
	language=octave,
	numbers=left,
	showstringspaces=true,
	stepnumber=1,
	tabsize=4,
	captionpos=b,
	title=\lstname,
	frame=single,
	float
}

\renewcommand{\listfigurename}{Spis ilustracji}
\renewcommand{\figurename}{Ilustracja}

\renewcommand{\thefootnote}{\alph{footnote}}
\begin{document}
\pagenumbering{roman}
\maketitle
\tableofcontents
\listoffigures
\listoftables
\clearpage
\setcounter{page}{1} \pagenumbering{arabic}
%%%%%%%%%%%%%%%%%%%%%%%%%%%%%%%%%%%%%%%%%%%%%%%%%%%%%%%%%%%%%%%%%%%%%%%%%%%%%%%%
\section{Zadanie}
Ćwiczenie polegało na zaprojektowaniu i zaimplementowaniu w Verilogu modułu dzielącego 8b dzielną przez 4b dzielnik (bez znaku).

TODO:
\begin{itemize}
	\item opis architektury: ctrl/data path
	\item graf stanów, parametryzacja, timingi
	\item opis resetu - async, potrzebny tylko raz (na starcie)
	\item raport syntezy - elimminacja ostrzeżeń o latchach, rozpoznanie fsm, czy zasoby się zgadzają
\end{itemize}
\section{Realizacja}

% TODO: zasoby
\begin{table}[!h]
	\centering
	\begin{tabular}{r||c|l}
		ctrl  & przerzutniki & logika sekwencyjna  \\
		snc   & przerzutniki & logika sekwencyjna  \\
		div   & przerzutniki & logika sekwencyjna  \\
		\hline\hline
		            & 8 wyjść   &  \\
	\end{tabular}
\end{table}

\clearpage
% TODO: opis modułu
\subsection{\texttt{ctrl}}
\subsubsection{Opis}
\begin{table}[!h]
	\begin{tabular}{r|c l}
		zadanie      & \multicolumn{2}{l}{Generuje pseudolosowy ciąg o okresie 7}\\
		architektura & \multicolumn{2}{l}{FSM}                                   \\
		wejścia      & FIXME       & FIXME                                       \\
		stan         & FIXME       & FIXME                                       \\
		wyjścia      & FIXME       & FIXME                                       \\
	\end{tabular}
\end{table}
\subsubsection{Implementacja}
\subsubsection{Synteza}
\clearpage
\section{Testowanie}
Mamy testbench z parametryzowalnym zegarem. Symulacja ,,Post translate'' nie wykryła żadnych problemów nawet dla zegara o częstotliwości 2ns, ,,Post-Place \& Route Simulation'' daje radę dla:
\begin{verbatim}
raport mowi. ze min. okres to 5.9costam, nasze pomiary spondzosci:
5.00 >> a = load (['test.dat']); sum(a(:,end-2:end) == 1)
        4096        4096        4096
4.50 >> a = load (['test.dat']); sum(a(:,end-2:end) == 1)
        4007        3964        4096
4.25 >> a = load (['test.dat']); sum(a(:,end-2:end) == 1)
        3606        3498        4096
4.00 >> a = load (['test.dat']); sum(a(:,end-2:end) == 1)
        2353        2325        4096
\end{verbatim}
\section{Wnioski}
\end{document}
